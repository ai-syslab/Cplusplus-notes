\documentclass{article}
\usepackage{listings}
\usepackage{xcolor}
\usepackage{amsmath}
\usepackage{hyperref}
\usepackage{enumitem}
\usepackage{geometry}
\usepackage{graphicx}
\usepackage[table]{xcolor}
\geometry{margin=1in}

\title{Control Flow}
\author{}
\date{}

\definecolor{codegray}{rgb}{0.5,0.5,0.5}
\definecolor{backcolour}{rgb}{0.95,0.95,0.92}

\lstdefinestyle{cppstyle}{
  backgroundcolor=\color{backcolour},
  commentstyle=\color{codegray},
  keywordstyle=\color{blue},
  numberstyle=\tiny\color{codegray},
  stringstyle=\color{red},
  basicstyle=\ttfamily\footnotesize,
  breakatwhitespace=false,
  breaklines=true,
  captionpos=b,
  keepspaces=true,
  numbers=none,
  numbersep=5pt,
  showspaces=false,
  showstringspaces=false,
  showtabs=false,
  tabsize=2,
  language=C++
}

\begin{document}

\maketitle


Control flow determines the order in which instructions are executed. It allows programs to make decisions, repeat actions, and manage complex logic.

\section{\texttt{if}, \texttt{else if}, and \texttt{else}}

Use when you want the program to take different actions depending on a condition.

\begin{lstlisting}[style=cppstyle]
int score = 85;
if (score >= 90) {
    cout << "Grade: A";
} else if (score >= 80) {
    cout << "Grade: B";
} else {
    cout << "Grade: C or below";
}
\end{lstlisting}

\section{\texttt{switch-case} Statement}

Use when checking a variable against fixed values (e.g., menu options).

\begin{lstlisting}[style=cppstyle]
int option = 2;
switch (option) {
    case 1:
        cout << "Start game";
        break;
    case 2:
        cout << "Load game";
        break;
    default:
        cout << "Invalid option";
}
\end{lstlisting}

Each case runs until a break is reached. Without break, the program continues (“falls through”) to the next case, which may cause unintended behavior.

\section{\texttt{while} Loop}

Use when you want to repeat something as long as a condition is true.

\begin{lstlisting}[style=cppstyle]
int count = 0;
while (count < 3) {
    cout << "Hello\n";
    count++;
}
\end{lstlisting}

\section{\texttt{do-while} Loop}

Use when you want to run the loop at least once.

\begin{lstlisting}[style=cppstyle]
int number;
do {
    cout << "Enter a positive number: ";
    cin >> number;
} while (number <= 0);
\end{lstlisting}

\texttt{cin} is used to take input from the user. It reads a value typed into the console and stores it in a variable.

\section{\texttt{for} Loop}

Use when the number of repetitions is known in advance.

\begin{lstlisting}[style=cppstyle]
for (int i = 1; i <= 5; i++) {
    cout << "Step " << i << "\n";
}
\end{lstlisting}

\subsection{Nested Loops (Without Arrays)}

Use when repeating something inside another repetition (e.g., patterns).

\begin{lstlisting}[style=cppstyle]
// Print a 3x3 square of stars
for (int row = 0; row < 3; row++) {
    for (int col = 0; col < 3; col++) {
        cout << "* ";
    }
    cout << "\n";
}
\end{lstlisting}

\subsection{Control Statements}

\texttt{break}, \texttt{continue}, and \texttt{return} help manage loop execution.

\begin{lstlisting}[style=cppstyle]
for (int i = 1; i <= 5; i++) {
    if (i == 3) continue;  // Skip 3
    if (i == 5) break;     // Stop at 4
    cout << i << " ";
}
\end{lstlisting}

\texttt{continue} skips the rest of the loop body and moves to the next iteration.

\texttt{break} exits the loop entirely, even if the condition is still true.

\subsection{Ternary Operator}

Short form of an \texttt{if-else} expression. A ternary operator is a compact way to choose between two values.

Syntax: \texttt{condition ? value-if-true : value-if-false}

\begin{lstlisting}[style=cppstyle]
int x = 10, y = 20;
int max = (x > y) ? x : y;
cout << "Max is " << max;
\end{lstlisting}

\subsection{Range-Based \texttt{for} Loop (C++11)}

Used to loop over collections like strings or vectors.

\begin{lstlisting}[style=cppstyle]
string name = "C++";
for (char ch : name) {
    cout << ch << "\n";
}
\end{lstlisting}

\end{document}