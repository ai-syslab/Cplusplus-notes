\documentclass{article}
\usepackage{amsmath}
\usepackage{listings}
\usepackage{xcolor}
\usepackage{graphicx}
\usepackage{geometry}
\geometry{margin=1in}

\title{Exceptions}
\author{}
\date{}

\definecolor{codegray}{rgb}{0.5,0.5,0.5}
\definecolor{backcolour}{rgb}{0.95,0.95,0.92}

\lstdefinestyle{cppstyle}{
  backgroundcolor=\color{backcolour},
  commentstyle=\color{codegray},
  keywordstyle=\color{blue},
  numberstyle=\tiny\color{codegray},
  stringstyle=\color{red},
  basicstyle=\ttfamily\footnotesize,
  breakatwhitespace=false,
  breaklines=true,
  captionpos=b,
  keepspaces=true,
  numbers=none,
  numbersep=5pt,
  showspaces=false,
  showstringspaces=false,
  showtabs=false,
  tabsize=2,
  language=C++
}

\begin{document}

\maketitle


\section{Overview}
\begin{itemize}
    \item Exceptions are used to handle runtime errors in C++.
    \item Control is transferred from the point of error to a handler.
    \item Main keywords: \texttt{try}, \texttt{catch}, \texttt{throw}.
\end{itemize}

\section{Basic Syntax}
\begin{lstlisting}[language=C++]
try {
    // Code that may throw an exception
    if (x == 0) {
        throw std::runtime_error("Division by zero");
    }
    result = y / x;
} catch (const std::exception& e) {
    std::cerr << "Error: " << e.what() << std::endl;
}
\end{lstlisting}

\section{Key Points}
\begin{itemize}
    \item \texttt{throw} raises an exception.
    \item \texttt{catch} defines how to handle it.
    \item Handlers can catch by type (e.g., \texttt{int}, \texttt{std::exception}).
    \item Uncaught exceptions cause program termination.
    \item Multiple \texttt{catch} blocks can be used for different types.
\end{itemize}

\section{Deallocation and Resource Management}
\begin{itemize}
    \item When an exception is thrown, C++ automatically destroys all local objects
          created since entering the \texttt{try} block.
    \item This process is called \textbf{stack unwinding}.
    \item Destructors of objects are called, so resources (memory, files, etc.)
          are released correctly.
    \item If raw pointers are used, memory leaks may occur unless managed carefully.
    \item Best practice: use RAII (Resource Acquisition Is Initialization) with
          smart pointers or resource-managing classes.
\end{itemize}

\section{Example: Automatic Deallocation}
\begin{lstlisting}[language=C++]
struct File {
    File(const char* name) { f = fopen(name, "r"); }
    ~File() { if (f) fclose(f); }  // destructor cleans up
    FILE* f;
};

try {
    File file("data.txt");  // allocated here
    throw std::runtime_error("Some error");
} catch (const std::exception& e) {
    std::cerr << "Caught: " << e.what() << std::endl;
}
// file is automatically closed by destructor
\end{lstlisting}

\section{Notes}
\begin{itemize}
    \item Prefer using standard exceptions from \texttt{<stdexcept>}.
    \item Exception handling separates error-handling code from normal logic.
    \item Use exceptions for exceptional cases, not regular control flow.
    \item RAII ensures resources are always freed, even when exceptions occur.
\end{itemize}


\end{document}