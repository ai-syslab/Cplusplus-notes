\documentclass{article}
\usepackage{amsmath}
\usepackage{listings}
\usepackage{xcolor}
\usepackage{graphicx}
\usepackage{geometry}
\geometry{margin=1in}

\title{Linked Lists}
\author{}
\date{}

\definecolor{codegray}{rgb}{0.5,0.5,0.5}
\definecolor{backcolour}{rgb}{0.95,0.95,0.92}

\lstdefinestyle{cppstyle}{
  backgroundcolor=\color{backcolour},
  commentstyle=\color{codegray},
  keywordstyle=\color{blue},
  numberstyle=\tiny\color{codegray},
  stringstyle=\color{red},
  basicstyle=\ttfamily\footnotesize,
  breakatwhitespace=false,
  breaklines=true,
  captionpos=b,
  keepspaces=true,
  numbers=none,
  numbersep=5pt,
  showspaces=false,
  showstringspaces=false,
  showtabs=false,
  tabsize=2,
  language=C++
}

\begin{document}

\maketitle


\section{Overview}

A \textbf{linked list} is a linear data structure where each element (called a \textit{node}) points to the next, forming a chain. Each node contains:
\begin{itemize}
  \item The actual data (here of type \texttt{int})
  \item A pointer to the next node
\end{itemize}

The \textbf{head pointer} refers to the first node of the list. It is essential for accessing the list. If the head is \texttt{nullptr}, the list is empty.

\subsection{Advantages over Arrays}
\begin{itemize}
  \item Dynamic size: no need to define capacity in advance.
  \item Efficient insertion/deletion: especially at the head or middle (no shifting required).
\end{itemize}

\subsection{Disadvantages}
\begin{itemize}
  \item Random access is not possible; traversal is sequential.
  \item Extra memory is used for storing pointers.
  \item Poorer cache performance compared to arrays.
\end{itemize}

\section{Visual Representation}
A simple linked list storing 3, 7, and 10 would look like:

\[
\texttt{head} \rightarrow 
\fbox{3\ $|$\ $\texttt{*}$} \rightarrow 
\fbox{7\ $|$\ $\texttt{*}$} \rightarrow 
\fbox{10\ $|$\ \texttt{NULL}}
\]

Each box represents a node storing an integer and a pointer to the next node.

\section{Node Structure and Linked List Class}

\begin{lstlisting}[style=cppstyle]


#ifndef LINKEDLIST_H
#define LINKEDLIST_H

struct Node {
    int data;
    Node* next;
    Node(int val);
};

class LinkedList {
private:
    Node* head;

public:
    LinkedList();
    ~LinkedList();

    void insertAtHead(int val);
    void insertAtEnd(int val);
    void deleteValue(int val);
    void print() const;
};

#endif

#include <iostream>
#include "LinkedList.h"

Node::Node(int val) : data(val), next(nullptr) {}

LinkedList::LinkedList() : head(nullptr) {}

LinkedList::~LinkedList() {
    Node* curr = head;
    while (curr) { // Until current node is null ptr
        Node* temp = curr;
        curr = curr->next;
        delete temp;
    }
}

void LinkedList::insertAtHead(int val) {
    Node* newNode = new Node(val);
    newNode->next = head;
    head = newNode;
}

void LinkedList::insertAtEnd(int val) {
    Node* newNode = new Node(val);
    if (!head) { // List empty
        head = newNode;
        return;
    }
    Node* temp = head;
    while (temp->next) // Until next node is nullptr
        temp = temp->next;
    temp->next = newNode;
}

void LinkedList::deleteValue(int val) {
    // If the list is empty, nothing to delete
    if (!head) return;

    // Check if the head node is the one to delete
    if (head->data == val) {
        Node* temp = head;        // Save the current head
        head = head->next;        // Move head to the next node
        delete temp;              // Delete the old head
        return;
    }

    // Traverse the list to find the node before the one to delete
    Node* curr = head;
    while (curr->next && curr->next->data != val) {
        curr = curr->next;
    }

    // If the node was found, delete it
    if (curr->next) {
        Node* temp = curr->next;              // Node to delete
        curr->next = curr->next->next;        // Bypass the node
        delete temp;                          // Free memory
    }
}

void LinkedList::print() const {
    Node* temp = head;
    while (temp) {
        std::cout << temp->data << " -> ";
        temp = temp->next;
    }
    std::cout << "NULL\n";
}

\end{lstlisting}

\subsection{Doubly Linked Lists}

A \textbf{doubly linked list} allows traversal in both directions. Each node has:
\begin{itemize}
  \item A data field
  \item A pointer to the next node
  \item A pointer to the previous node
\end{itemize}

\begin{lstlisting}[style=cppstyle]
struct DNode {
    int data;
    DNode* prev;
    DNode* next;
    DNode(int val) : data(val), prev(nullptr), next(nullptr) {}
};
\end{lstlisting}

\subsection{Templated Linked List}

A generic linked list using templates:

\begin{lstlisting}[style=cppstyle]
template <typename T>
struct Node {
    T data;
    Node* next;
    Node(T val) : data(val), next(nullptr) {}
};
\end{lstlisting}

This allows storing any data type: \texttt{int}, \texttt{double}, \texttt{std::string}, or user-defined objects, with the same logic as above reused across types.
\end{document}