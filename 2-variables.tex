\documentclass{article}
\usepackage{listings}
\usepackage{xcolor}
\usepackage{amsmath}
\usepackage{hyperref}
\usepackage{enumitem}
\usepackage{geometry}
\usepackage{graphicx}
\usepackage[table]{xcolor}
\geometry{margin=1in}

\title{Variables and Operators}
\author{}
\date{}

\definecolor{codegray}{rgb}{0.5,0.5,0.5}
\definecolor{backcolour}{rgb}{0.95,0.95,0.92}

\lstdefinestyle{cppstyle}{
  backgroundcolor=\color{backcolour},
  commentstyle=\color{codegray},
  keywordstyle=\color{blue},
  numberstyle=\tiny\color{codegray},
  stringstyle=\color{red},
  basicstyle=\ttfamily\footnotesize,
  breakatwhitespace=false,
  breaklines=true,
  captionpos=b,
  keepspaces=true,
  numbers=none,
  numbersep=5pt,
  showspaces=false,
  showstringspaces=false,
  showtabs=false,
  tabsize=2,
  language=C++
}

\begin{document}

\maketitle

\section{How a Computer is Organized}

A computer is made up of several key components that work together to execute programs like your C++ code.

\subsection*{Main Components}

\renewcommand{\arraystretch}{1.5} % Increase row height
\setlength{\tabcolsep}{12pt}      % Increase column padding

\begin{center}
\begin{tabular}{|l|p{9cm}|}
\hline
\rowcolor{lightgray} \textbf{Component} & \textbf{Description} \\
\hline
\textbf{CPU} & The Central Processing Unit is the brain of the computer. It performs calculations and executes instructions. \\
\hline
\textbf{RAM} & Random Access Memory is short-term memory. It temporarily holds data and instructions that are in use. \\
\hline
\textbf{Storage} & Long-term memory (e.g., SSD or HDD). It permanently stores files, programs, and the operating system. \\
\hline
\textbf{Input Devices} & Devices like the keyboard, mouse, or microphone that allow users to send data into the computer. \\
\hline
\textbf{Output Devices} & Devices such as monitors, speakers, or printers that present data from the computer to the user. \\
\hline
\end{tabular}
\end{center}

\subsection*{Basic Program Flow}
\begin{itemize}
    \item You write C++ code as high-level instructions.
    \item The compiler converts this code into machine code.
    \item The CPU executes the machine code, using RAM to store data temporarily.
    \item Input/output devices allow the user to interact with the program.
\end{itemize}

\section{What Are Variables in C++?}

A \textbf{variable} is a named location in memory that stores data.



\begin{figure}[h]
    \centering
    \includegraphics[width=0.2\textwidth]{figures/var.png}  % Adjust width as needed
    \caption{Illustration of variables in memory. Here x is a variable with value 28.}
    \label{fig:var}
\end{figure}


\subsection*{Common Data Types}

Each variable must have a ``type" that specifies the type of data it holds. For example, int holds integer data and double holds real numbers.

\begin{center}
\begin{tabular}{|l|l|l|}
\hline
\textbf{Type} & \textbf{Meaning} & \textbf{Example} \\
\hline
\texttt{int} & Integer & \texttt{int age = 25;} \\
\texttt{float} & Decimal number & \texttt{float temp = 98.6;} \\
\texttt{double} & More precise decimal & \texttt{double pi = 3.14159;} \\
\texttt{char} & Single character & \texttt{char grade = 'A';} \\
\texttt{bool} & True or false & \texttt{bool passed = true;} \\
\hline
\end{tabular}
\end{center}

\subsection*{Constant Variables with \texttt{const}}

Sometimes you may want to declare a variable whose value should not change after initialization. In such cases, use the \texttt{const} keyword. This makes the variable read-only.

\begin{lstlisting}[style=cppstyle]
const double pi = 3.14159;
\end{lstlisting}

Attempting to modify a \texttt{const} variable later in the program will result in a compilation error. This is useful for defining fixed values like mathematical constants, configuration settings, or immutable identifiers.

\textbf{Note:} You must initialize a \texttt{const} variable at the time of declaration.

\subsection*{Rules for Naming Variables}
\begin{itemize}
    \item Must start with a letter or underscore (\texttt{\_})
    \item Cannot use C++ keywords (e.g., \texttt{int}, \texttt{while})
    \item Case-sensitive (\texttt{score} and \texttt{Score} are different)
\end{itemize}

\section{What Are Operators?}

Operators are symbols that tell the computer to perform specific operations on \textbf{variables} and \textbf{values}.

\section*{Operators in C++}

Below is a summary of common types of operators in C++:

\renewcommand{\arraystretch}{1.5}
\setlength{\tabcolsep}{10pt}

\subsection*{Arithmetic Operators}

\begin{center}
\begin{tabular}{|c|l|c|c|}
\hline
\textbf{Operator} & \textbf{Use} & \textbf{Example} & \textbf{Result} \\
\hline
\texttt{+} & Addition & \texttt{5 + 3} & \texttt{8} \\
\hline
\texttt{-} & Subtraction & \texttt{5 - 3} & \texttt{2} \\
\hline
\texttt{*} & Multiplication & \texttt{5 * 3} & \texttt{15} \\
\hline
\texttt{/} & Division & \texttt{5 / 2} & \texttt{2} (integer division) \\
\hline
\texttt{\%} & Modulus (remainder) & \texttt{5 \% 2} & \texttt{1} \\
\hline
\end{tabular}
\end{center}

\subsection*{Assignment Operators}

\begin{center}
\begin{tabular}{|c|l|c|}
\hline
\textbf{Operator} & \textbf{Use} & \textbf{Example} \\
\hline
\texttt{=} & Assign value & \texttt{x = 10;} \\
\hline
\texttt{+=} & Add and assign & \texttt{x += 5;} (\texttt{x = x + 5}) \\
\hline
\texttt{-=} & Subtract and assign & \texttt{x -= 2;} \\
\hline
\texttt{*=} & Multiply and assign & \texttt{x *= 3;} \\
\hline
\end{tabular}
\end{center}

\subsection*{Comparison Operators}

\begin{center}
\begin{tabular}{|c|l|c|}
\hline
\textbf{Operator} & \textbf{Meaning} & \textbf{Example} \\
\hline
\texttt{==} & Equal to & \texttt{x == 5} \\
\hline
\texttt{!=} & Not equal to & \texttt{x != 5} \\
\hline
\texttt{<}, \texttt{>}, \texttt{<=}, \texttt{>=} & Less than, etc. & \texttt{x < 10} \\
\hline
\end{tabular}
\end{center}

\subsection*{Logical Operators}

\begin{center}
\begin{tabular}{|c|l|c|}
\hline
\textbf{Operator} & \textbf{Meaning} & \textbf{Example} \\
\hline
\texttt{\&\&} & AND (both conditions true) & \texttt{x > 0 \&\& y > 0} \\
\hline
\texttt{||} & OR (at least one true) & \texttt{x > 0 || y > 0} \\
\hline
\texttt{!} & NOT (negation) & \texttt{!is\_valid} \\
\hline
\end{tabular}
\end{center}

\section{How Operators and Variables Work Together}

Variables store values.  
Operators act on these values to compute new results, which can then be stored back in variables.

\begin{verbatim}
int a = 5;
int b = 2;
int result = a + b;  // '+' is the operator acting on variables a and b
\end{verbatim}

In this case:
\begin{itemize}
    \item \texttt{a} and \texttt{b} are \textbf{variables}
    \item \texttt{+} is an \textbf{arithmetic operator}
    \item The result (7) is stored in another variable \texttt{result}
\end{itemize}

\section{Analogy Table}

\begin{center}
\begin{tabular}{|l|l|l|}
\hline
\textbf{Concept} & \textbf{Role in C++} & \textbf{Analogy} \\
\hline
Variable & A box with a name that stores data & A labeled jar \\
Value & The contents inside the variable & Candy inside the jar \\
Operator & A tool used to work with values & A spoon to add/subtract candy \\
\hline
\end{tabular}
\end{center}

\section{Operator Actions on Variables}

\begin{center}
\begin{tabular}{|l|l|}
\hline
\textbf{Expression} & \textbf{Action Performed} \\
\hline
\texttt{x = 10} & Store 10 in variable \texttt{x} \\
\texttt{y = x + 5} & Add \texttt{x} and 5, store in \texttt{y} \\
\texttt{x += 3} & Increase \texttt{x} by 3 \\
\texttt{z = x * y} & Multiply \texttt{x} and \texttt{y}, store in \texttt{z} \\
\hline
\end{tabular}
\end{center}

\noindent\textbf{In short:} Operators manipulate the values stored in variables to produce new results.

\begin{lstlisting}[style=cppstyle]
#include <iostream>
using namespace std;

int main() {
    // Variable declarations
    int a = 10;
    int b = 3;

    // Arithmetic operators
    int sum = a + b;
    int difference = a - b;
    int product = a * b;
    int quotient = a / b;
    int remainder = a % b;

    // Output results
    cout << "a = " << a << ", b = " << b << endl;
    cout << "Sum (a + b) = " << sum << endl;
    cout << "Difference (a - b) = " << difference << endl;
    cout << "Product (a * b) = " << product << endl;
    cout << "Quotient (a / b) = " << quotient << endl;
    cout << "Remainder (a % b) = " << remainder << endl;

    // Relational operator
    cout << "Is a greater than b? " << (a > b) << endl;

    // Logical operator
    bool result = (a > 5) && (b < 5);
    cout << "Is a > 5 AND b < 5? " << result << endl;

    return 0;
}
\end{lstlisting}

Comple and run using - 
\begin{verbatim}
g++ demo.cpp -o demo
./demo  
\end{verbatim}


\end{document}