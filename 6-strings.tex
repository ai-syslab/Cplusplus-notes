\documentclass{article}
\usepackage{listings}
\usepackage{xcolor}
\usepackage{amsmath}
\usepackage{hyperref}
\usepackage{enumitem}
\usepackage{geometry}
\usepackage{graphicx}
\usepackage[table]{xcolor}
\usepackage{tikz}
\usetikzlibrary{positioning}
\geometry{margin=1in}

\title{Strings}
\author{}
\date{}

\definecolor{codegray}{rgb}{0.5,0.5,0.5}
\definecolor{backcolour}{rgb}{0.95,0.95,0.92}

\lstdefinestyle{cppstyle}{
  backgroundcolor=\color{backcolour},
  commentstyle=\color{codegray},
  keywordstyle=\color{blue},
  numberstyle=\tiny\color{codegray},
  stringstyle=\color{red},
  basicstyle=\ttfamily\footnotesize,
  breakatwhitespace=false,
  breaklines=true,
  captionpos=b,
  keepspaces=true,
  numbers=none,
  numbersep=5pt,
  showspaces=false,
  showstringspaces=false,
  showtabs=false,
  tabsize=2,
  language=C++
}

\begin{document}

\maketitle


C++ provides support for strings using two main approaches: C-style strings and the \texttt{std::string} class from the Standard Library. The \texttt{std::string} class is preferred for most applications due to its safety and flexibility.

\section{C-style Strings}

A C-style string is an array of characters terminated by the null character \texttt{\textbackslash0}. It is declared using a character array.

\begin{lstlisting}[style=cppstyle]
char greeting[] = "Hello";
\end{lstlisting}

\texttt{cstring} library functions like \texttt{strlen()}, \texttt{strcpy()}, and \texttt{strcmp()} are used to manipulate C-style strings.

\section{std::string Class}

The \texttt{std::string} class is part of the \texttt{<string>} header and provides a wide range of operations.

\begin{itemize}
  \item Safe and dynamic resizing.
  \item Supports operators like \texttt{+} for concatenation and \texttt{==} for comparison.
  \item Can use member functions like \texttt{length()}, \texttt{substr()}, \texttt{find()}, and \texttt{append()}.
\end{itemize}

\begin{lstlisting}[style=cppstyle]
#include <iostream>
#include <string>
using namespace std;

int main() {
    string name = "Alice";
    string greeting = "Hello, " + name;
    cout << greeting << endl;
    return 0;
}
\end{lstlisting}

\section{Common Operations}

\begin{lstlisting}[style=cppstyle]
string s = "example";
int len = s.length();        // Get length
char c = s[0];               // Access character
s += " string";              // Append
s.substr(2, 3);              // "amp"
s.find("amp");               // Returns 2
\end{lstlisting}

\end{document}