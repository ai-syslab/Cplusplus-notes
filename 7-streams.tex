\documentclass{article}
\usepackage{listings}
\usepackage{xcolor}
\usepackage{amsmath}
\usepackage{hyperref}
\usepackage{enumitem}
\usepackage{geometry}
\usepackage{graphicx}
\usepackage[table]{xcolor}
\usepackage{tikz}
\usetikzlibrary{positioning}
\geometry{margin=1in}

\title{Stream Operations}
\author{}
\date{}

\definecolor{codegray}{rgb}{0.5,0.5,0.5}
\definecolor{backcolour}{rgb}{0.95,0.95,0.92}

\lstdefinestyle{cppstyle}{
  backgroundcolor=\color{backcolour},
  commentstyle=\color{codegray},
  keywordstyle=\color{blue},
  numberstyle=\tiny\color{codegray},
  stringstyle=\color{red},
  basicstyle=\ttfamily\footnotesize,
  breakatwhitespace=false,
  breaklines=true,
  captionpos=b,
  keepspaces=true,
  numbers=none,
  numbersep=5pt,
  showspaces=false,
  showstringspaces=false,
  showtabs=false,
  tabsize=2,
  language=C++
}

\begin{document}

\maketitle


C++ uses streams to perform input and output (I/O) operations. A stream is an abstraction that represents a device where data can be read from or written to, such as the console, a file, or a string.

\section{Standard Input and Output}

\texttt{cin} and \texttt{cout} are used for reading from standard input and writing to standard output respectively.

\begin{lstlisting}[style=cppstyle]
#include <iostream>
using namespace std;

int main() {
    int age;
    cout << "Enter your age: ";
    cin >> age;
    cout << "You are " << age << " years old." << endl;
    return 0;
}
\end{lstlisting}

\section{String Streams}

The \texttt{sstream} header provides stream classes that operate on strings: \texttt{istringstream}, \texttt{ostringstream}, and \texttt{stringstream}.

\begin{itemize}
  \item \texttt{istringstream} is used to extract data from strings.
  \item \texttt{ostringstream} is used to write data to a string.
  \item \texttt{stringstream} can do both.
\end{itemize}

\begin{lstlisting}[style=cppstyle]
#include <iostream>
#include <sstream>
using namespace std;

int main() {
    string data = "42 3.14 hello";
    istringstream iss(data);
    int x;
    double y;
    string word;
    iss >> x >> y >> word;
    cout << x << ", " << y << ", " << word << endl;
    return 0;
}
\end{lstlisting}

\section{File Streams}

The \texttt{fstream} header provides \texttt{ifstream} for reading files and \texttt{ofstream} for writing files.

\begin{lstlisting}[style=cppstyle]
#include <iostream>
#include <fstream>
using namespace std;

int main() {
    ofstream outFile("output.txt");
    outFile << "Hello, file!" << endl;
    outFile.close();

    ifstream inFile("output.txt");
    string line;
    getline(inFile, line);
    cout << "Read from file: " << line << endl;
    inFile.close();
    return 0;
}
\end{lstlisting}

\section{Notes}
\begin{itemize}
  \item Always check if file streams are successfully opened using \texttt{is\_open()} or by checking the stream object in a condition.
  \item Use \texttt{getline()} for reading lines that may include spaces.
  \item Use \texttt{eof()} or loop conditions like \texttt{while(inFile >> var)} to process streams until the end.
\end{itemize}

\section{A more detailed example}
This example demonstrates writing structured data to a file and then reading it back line by line.

\begin{lstlisting}[style=cppstyle]
#include <iostream>
#include <fstream>
#include <vector>
using namespace std;

int main() {
    // Write to file
    ofstream outFile("students.txt");
    if (!outFile) {
        cerr << "Failed to open file for writing." << endl;
        return 1;
    }

    vector<string> students = {
        "101 Alice 88.5",
        "102 Bob 92.0",
        "103 Carol 79.5"
    };

    for (const auto& line : students) {
        outFile << line << endl;
    }
    outFile.close();

    // Read from file
    ifstream inFile("students.txt");
    if (!inFile) {
        cerr << "Failed to open file for reading." << endl;
        return 1;
    }

    string line;
    while (getline(inFile, line)) {
        istringstream iss(line);
        int id;
        string name;
        float score;
        iss >> id >> name >> score;
        cout << "ID: " << id << ", Name: " << name << ", Score: " << score << endl;
    }
    inFile.close();

    return 0;
}
\end{lstlisting}

\textbf{Explanation:} 
\begin{itemize}
  \item The first part writes a vector of student records to a file using \texttt{ofstream}.
  \item The second part reads each line using \texttt{getline()}, then parses each line with \texttt{istringstream}.
  \item The \texttt{auto} keyword in the range-based \texttt{for} loop lets the compiler deduce the type of each element in the vector. Here, each element is a \texttt{std::string}.
  \item \texttt{cerr} is used to print error messages to the standard error stream. It is unbuffered, meaning output is sent directly to the terminal without being stored temporarily in a buffer. This ensures that error messages appear immediately.
\end{itemize}

\end{document}