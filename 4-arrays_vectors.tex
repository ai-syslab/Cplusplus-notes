\documentclass{article}
\usepackage{listings}
\usepackage{xcolor}
\usepackage{amsmath}
\usepackage{hyperref}
\usepackage{enumitem}
\usepackage{geometry}
\usepackage{graphicx}
\usepackage[table]{xcolor}
\geometry{margin=1in}

\title{Arrays and Vectors}
\author{}
\date{}

\definecolor{codegray}{rgb}{0.5,0.5,0.5}
\definecolor{backcolour}{rgb}{0.95,0.95,0.92}

\lstdefinestyle{cppstyle}{
  backgroundcolor=\color{backcolour},
  commentstyle=\color{codegray},
  keywordstyle=\color{blue},
  numberstyle=\tiny\color{codegray},
  stringstyle=\color{red},
  basicstyle=\ttfamily\footnotesize,
  breakatwhitespace=false,
  breaklines=true,
  captionpos=b,
  keepspaces=true,
  numbers=none,
  numbersep=5pt,
  showspaces=false,
  showstringspaces=false,
  showtabs=false,
  tabsize=2,
  language=C++
}

\begin{document}

\maketitle


Arrays and vectors let you store multiple values in a single variable. They are useful when dealing with groups of data—like a list of scores or names.

\section{Arrays}

An array is a fixed-size collection of elements of the same type. Each element is accessed by an index (starting from 0).

\begin{lstlisting}[style=cppstyle]
// Declare and initialize an array of size 5
int scores[5] = {90, 85, 78, 92, 88};

// Access and modify elements
cout << "First score: " << scores[0] << endl;
scores[1] = 95;
\end{lstlisting}

\textbf{Key Points:}
\begin{itemize}
    \item Array size must be known at compile time.
    \item Indices start from 0 and go up to size - 1.
    \item Accessing an invalid index leads to undefined behavior.
\end{itemize}

\textbf{Memory Layout:} \\
An array stores elements in \textbf{contiguous memory locations}. This means if \texttt{scores[0]} is at address \texttt{1000} and each \texttt{int} takes 4 bytes, then:

\begin{itemize}
    \item \texttt{scores[1]} is at address \texttt{1004}
    \item \texttt{scores[2]} is at address \texttt{1008}
    \item and so on
\end{itemize}

This layout allows fast access to any element using its index and is important for performance in loops.


\section{Looping Over Arrays}

Arrays work well with loops to process each element.

\begin{lstlisting}[style=cppstyle]
for (int i = 0; i < 5; i++) {
    cout << "Score " << i << ": " << scores[i] << endl;
}
\end{lstlisting}

\section{Multidimensional Arrays}

You can have arrays of arrays, useful for tables or grids.

\begin{lstlisting}[style=cppstyle]
int matrix[2][3] = {
    {1, 2, 3},
    {4, 5, 6}
};

cout << matrix[1][2];  // Outputs 6
\end{lstlisting}

\section{Vectors (C++ Standard Library)}

A \texttt{vector} is a dynamic array. Unlike regular arrays, vectors can grow or shrink in size during runtime.

To use vectors, include the header:
\begin{lstlisting}[style=cppstyle]
#include <vector>
using namespace std;
\end{lstlisting}

\textbf{Basic Usage:}
\begin{lstlisting}[style=cppstyle]
vector<int> numbers;         // empty vector
numbers.push_back(10);       // add 10
numbers.push_back(20);
cout << numbers[0];          // outputs 10
\end{lstlisting}

\texttt{<int>} means this vector will store elements of type \texttt{int}. You can replace \texttt{int} with other types like \texttt{double}, \texttt{char}, or \texttt{string}.


\textbf{Accessing Elements Safely:}
\begin{lstlisting}[style=cppstyle]
cout << numbers.at(1); // safer than numbers[1], checks bounds
\end{lstlisting}
The method \texttt{at(index)} throws an exception if the index is out of bounds, unlike the \texttt{[]} operator.

\textbf{Removing the Last Element:}
\begin{lstlisting}[style=cppstyle]
numbers.pop_back(); // removes the last element
\end{lstlisting}
This is useful when treating the vector like a stack (LIFO).

\textbf{Understanding Iterators:}

An iterator is like a pointer that allows you to traverse through elements in a container (like a vector). The begin() method returns an iterator pointing to the first element, and end() returns an iterator one past the last element.

\begin{lstlisting}[style=cppstyle]
for (vector::iterator it = numbers.begin(); it != numbers.end(); ++it) {
    cout << *it << " ";
}
\end{lstlisting}

Here, \texttt{*it} accesses the value pointed to by the iterator.

\textbf{Erasing Elements:}
\begin{lstlisting}[style=cppstyle]
numbers.erase(numbers.begin() + 1); // removes the second element
\end{lstlisting}
The \texttt{erase()} function removes an element or a range of elements. You provide it an iterator pointing to the element(s) to remove.

\textbf{Inserting Elements:}
\begin{lstlisting}[style=cppstyle]
numbers.insert(numbers.begin() + 1, 15); // insert 15 at index 1
\end{lstlisting}
The \texttt{insert()} function adds an element at a specific position, again using an iterator.

\textbf{Example - Full Workflow:}
\begin{lstlisting}[style=cppstyle]
vector numbers = {10, 20, 30};
numbers.insert(numbers.begin() + 1, 15); // 10, 15, 20, 30
numbers.erase(numbers.begin() + 2);      // 10, 15, 30
numbers.pop_back();                      // 10, 15
for (int n : numbers) {
    cout << *it << " ";
}
\end{lstlisting}
\textbf{Looping with Range-Based for (C++11):}
\begin{lstlisting}[style=cppstyle]
for (int n : numbers) {
    cout << n << " ";
}
\end{lstlisting}

\textbf{Why Vectors?}
\begin{itemize}
    \item Size can change at runtime.
    \item Provides useful functions like \texttt{push\_back()}, \texttt{size()}, and \texttt{clear()}.
    \item Safer and more flexible than raw arrays.
\end{itemize}

\end{document}