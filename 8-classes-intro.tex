\documentclass{article}
\usepackage{listings}
\usepackage{xcolor}
\usepackage{amsmath}
\usepackage{hyperref}

\title{Introduction to C++ Classes}
\author{}
\date{}

\definecolor{codegray}{rgb}{0.5,0.5,0.5}
\definecolor{backcolour}{rgb}{0.95,0.95,0.92}

\lstdefinestyle{cppstyle}{
  backgroundcolor=\color{backcolour},
  commentstyle=\color{codegray},
  keywordstyle=\color{blue},
  numberstyle=\tiny\color{codegray},
  stringstyle=\color{red},
  basicstyle=\ttfamily\footnotesize,
  breakatwhitespace=false,
  breaklines=true,
  captionpos=b,
  keepspaces=true,
  numbers=none,
  numbersep=5pt,
  showspaces=false,
  showstringspaces=false,
  showtabs=false,
  tabsize=2,
  language=C++
}

\begin{document}

\maketitle

\section{What are Classes in C++?}
A class in C++ is a blueprint for creating objects. It encapsulates data and functions that operate on that data. Classes are a fundamental feature of object-oriented programming (OOP).

\section{Why Use Classes?}
Classes help in organizing complex programs by bundling data and the operations on that data together. This promotes:
\begin{itemize}
    \item Modularity: Classes allow you to group related data and functions together, making code easier to manage and organize.
	\item Code reuse: Once defined, a class can be reused across programs and extended through inheritance to avoid rewriting code.
	\item Encapsulation: Classes enable you to hide internal details and protect object state by restricting direct access to data.
	\item Abstraction: Classes let you define high-level interfaces while hiding complex implementation details from the user.
\end{itemize}

\section{Data and Methods}
A class contains:
\begin{itemize}
  \item \textbf{Data members} (variables to store state)
  \item \textbf{Member functions} or \textbf{methods} (functions to define behavior)
\end{itemize}

\section{Encapsulation}
Encapsulation means restricting direct access to some of an object's components. This is done using access specifiers:
\begin{itemize}
  \item \texttt{public}: accessible from outside the class
  \item \texttt{private}: accessible only from within the class
  \item \texttt{protected}: accessible in derived classes
\end{itemize}

\section{Constructors}
Constructors are special member functions called automatically when an object is created. They initialize the object's data members.

\section{Getters and Setters}
Getters retrieve the value of private data members. Setters allow controlled modification of private data members.

\section{Example Code}
\begin{lstlisting}[style=cppstyle]
#include <iostream>
using namespace std;

class Car {
private:
    string brand;
    int year;

public:
    // Constructor
    Car(string b, int y) {
        brand = b;
        year = y;
    }

    // Getter for brand
    string getBrand() {
        return brand;
    }

    // Setter for brand
    void setBrand(string b) {
        brand = b;
    }

    // Getter for year
    int getYear() {
        return year;
    }

    // Setter for year
    void setYear(int y) {
        year = y;
    }

    // Method to display car info
    void displayInfo() {
        cout << "Brand: " << brand << ", Year: " << year << endl;
    }
};

int main() {
    Car myCar("Toyota", 2020);
    myCar.displayInfo();
    
    myCar.setBrand("Honda");
    myCar.setYear(2022);
    myCar.displayInfo();

    return 0;
}
\end{lstlisting}

\section{Operator Overloading}

Operator overloading allows you to redefine how operators work with user-defined types (e.g., classes). This enables natural syntax when working with custom objects like fractions, complex numbers, vectors, etc.


\begin{lstlisting}[style=cppstyle]
	#include <iostream>

class Fraction {
    private:
		int numerator;
		int denominator;
	
	public:
		Fraction(int num = 0, int den = 1) : numerator(num), denominator(den) {}
	
		// Overload +
		Fraction operator+(const Fraction& other) const {
			int num = numerator * other.denominator + other.numerator * denominator;
			int den = denominator * other.denominator;
			return Fraction(num, den);
		}
	
		// Overload -
		Fraction operator-(const Fraction& other) const {
			int num = numerator * other.denominator - other.numerator * denominator;
			int den = denominator * other.denominator;
			return Fraction(num, den);
		}
	
		void display() const {
			std::cout << numerator << "/" << denominator << "\n";
		}
};
	
	// Demo
	int main() {
		Fraction f1(1, 3), f2(1, 6);
		Fraction sum = f1 + f2;
		Fraction diff = f1 - f2;
	
		std::cout << "Sum: ";
		sum.display();
		std::cout << "Difference: ";
		diff.display();
	
		return 0;
	}
\end{lstlisting}


\section{Conclusion}
Classes are essential for building scalable and maintainable C++ applications. They provide structure and abstraction through data hiding and encapsulation.

\end{document}
